%%%%%%%%%%%%%%%%%%%%%%%%%%%%%%%%%%%%%%%%%%%%%%%%%%%%%%%%%%%%%%%%%%%%%%%
%%%%%%%%%%%%%%%%%%%%%%%%%%%%%%%%%%%%%%%%%%%%%%%%%%%%%%%%%%%%%%%%%%%%%%%
%%%%%                                                                 %
%%%%%     preamble.tex                                                %
%%%%%                                                                 %
%%%%% Author:      Michael Muehlberghuber                             %
%%%%% Created:     01.07.2012                                         %
%%%%% Description: This file contains the preamble of the             %
%%%%%              Semester-/Master-Project LaTeX report example.     %
%%%%%                                                                 %
%%%%% History:                                                        %
%%%%%%%%%%%%%%                                                        %
%%%%% 2012/07/01:  *) Created initial version.                        %
%%%%%                                                                 %
%%%%%%%%%%%%%%%%%%%%%%%%%%%%%%%%%%%%%%%%%%%%%%%%%%%%%%%%%%%%%%%%%%%%%%%
%%%%%%%%%%%%%%%%%%%%%%%%%%%%%%%%%%%%%%%%%%%%%%%%%%%%%%%%%%%%%%%%%%%%%%%

%%%%%%%%%%%%%%%%%%%%%%%%%%%%%%%%%%%%%%%%%%%%%%%%%%%%%%%%%%%%%%%%%%%%%%%
%%%%%                                                                 %
%%%%%     Package Loading                                             %
%%%%%                                                                 %
%%%%%%%%%%%%%%%%%%%%%%%%%%%%%%%%%%%%%%%%%%%%%%%%%%%%%%%%%%%%%%%%%%%%%%%

% Determines the input encoding.
\usepackage[%
 utf8,
% latin1
]{inputenc}

% ---------------------------------------------------------------------

% Determines the output encoding.
\usepackage[T1]{fontenc}

% ---------------------------------------------------------------------

% Determines language settings.
\usepackage[%
 english    % You may change this to 'ngerman' in order to write a
            % german report.
]{babel}

% ---------------------------------------------------------------------

% Provides image loading.
\usepackage{graphicx}

% ---------------------------------------------------------------------

% Provides customization of chapter headings.
\usepackage[%
	Lenny     % Choose a nice layout for chapter headings.
]{fncychap}

% ---------------------------------------------------------------------

% Provides some blindtext.
\usepackage{lipsum}

% ---------------------------------------------------------------------

% Provides stretchable tables.
\usepackage{tabularx}

% ---------------------------------------------------------------------

% Provides some fancy boxes.
\usepackage{fancybox}

% ---------------------------------------------------------------------

% Provides subfigures.
\usepackage{subcaption}

% ---------------------------------------------------------------------

% Provides colors in LaTeX.
\usepackage{xcolor}

% ---------------------------------------------------------------------

% Provides conditionals (for titlepage).
\usepackage{xifthen}

% ---------------------------------------------------------------------

% Provides the algorithm environment
\usepackage[ruled,%
            linesnumbered]{algorithm2e}

% ---------------------------------------------------------------------

% Provides bold greek math symbols.
\usepackage{bm}

% ---------------------------------------------------------------------

% Allows to include pdf documents.
\usepackage{pdfpages}

% ---------------------------------------------------------------------

% Provides nicer tables than the standard tables.
\usepackage{booktabs}

% ---------------------------------------------------------------------

% Provides simple line spacings.
\usepackage{setspace}

% ---------------------------------------------------------------------

% Provides simple line spacings.
\usepackage{geometry}

% ---------------------------------------------------------------------

% Provides more customizeable captions.
\usepackage{capt-of}

% ---------------------------------------------------------------------

% Provides small table of contents (e.g., for single chapters or the
% appendix).
\usepackage{minitoc}

% ---------------------------------------------------------------------

% Provides a simple command to describe a directory tree.
\usepackage{dirtree}

% ---------------------------------------------------------------------

%%%%%                                                             %%%%%
%%%%% ATTENTION: Loading further packagaes should go in here.     %%%%%
%%%%%                                                             %%%%%

% ---------------------------------------------------------------------

% Provides hyperlinks within your document. Should always be loaded at
% the end.
\usepackage{hyperref}

% ---------------------------------------------------------------------

% Provides multiple glossaries (incl. list acronyms, list of symbols,
% etc.).
\usepackage[%
 toc,              % Add the glossaries to the table of contents.
 acronym,          % Add a list of acronyms.
 section=chapter,  % Show glossary headers as chapters.
 nonumberlist,     % Do not print the page numbers next to glossary
                   % entries.
]{glossaries}


\usepackage{float}
\usepackage{amsmath}
\usepackage{threeparttable}
\usepackage{fancyvrb}
\usepackage{listings}
\usepackage{pdflscape}
% Tikz/plots
\usepackage{tikz}
\usetikzlibrary{backgrounds}
\usepackage{pgfplots}
\usetikzlibrary{pgfplots.groupplots}

\pgfplotscreateplotcyclelist{mylist}{ 
  fill=gray!20,draw=black!90,line width=.2pt\\%
  fill=blue!60,draw=black!90,line width=.2pt\\%
  fill=red!60,draw=black!90,line width=.2pt\\%
  fill=orange!60,draw=black!90,line width=.2pt\\%
  fill=yellow!60,draw=black!90,line width=.2pt\\%
  fill=green!60,draw=black!90,line width=.2pt\\%
  fill=brown!60,draw=black!90,line width=.2pt\\%
  fill=black!60,draw=black!90,line width=.2pt\\%
  fill=white\\%
}
\newcommand\drawbar[1]{%
  \begin{tikzpicture}%
    \draw[fill=#1,draw=black!90,ultra thin] (0pt,0pt) rectangle (3pt,8pt);%
    \draw[fill=#1,draw=black!90,ultra thin] (6pt,0pt) rectangle (9pt,6pt);%
  \end{tikzpicture}\hspace{0pt}%
}


%%%%%%%%%%%%%%%%%%%%%%%%%%%%%%%%%%%%%%%%%%%%%%%%%%%%%%%%%%%%%%%%%%%%%%%
%%%%%                                                                 %
%%%%%     Custom Settings                                             %
%%%%%                                                                 %
%%%%%%%%%%%%%%%%%%%%%%%%%%%%%%%%%%%%%%%%%%%%%%%%%%%%%%%%%%%%%%%%%%%%%%%
% Do not use sans-serif fonts for all dispositions (chapters,
% sections, etc.)
\setkomafont{disposition}{\normalfont\bfseries}


%%%%%%%%%%%%%%%%%%%%%%%%%%%%%%%%%%%%%%%%%%%%%%%%%%%%%%%%%%%%%%%%%%%%%%%
%%%%%                                                                 %
%%%%%     Custom Macros                                               %
%%%%%                                                                 %
%%%%%%%%%%%%%%%%%%%%%%%%%%%%%%%%%%%%%%%%%%%%%%%%%%%%%%%%%%%%%%%%%%%%%%%
% Create an inline command for shell commands.
\newcommand{\shell}[1]{\texttt{#1}}

% Create an enviroment for a shell commands.
\newenvironment{shellenv}%
{\VerbatimEnvironment%
 \begin{Sbox}\begin{minipage}{0.97\textwidth}\begin{Verbatim}%
}%
{\end{Verbatim}\end{minipage}\end{Sbox}%
\setlength{\fboxsep}{6pt}\shadowbox{\TheSbox}}%

% Create an inline command for files.
\newcommand{\file}[1]{\texttt{#1}}

% Create a command for command parameters.
\newcommand{\parameter}[1]{$<$#1$>$}

\newcommand{\instr}[1]{\texttt{#1}}


\definecolor{lightGray}{RGB}{240,240,240}

\lstnewenvironment{instrenv}{\lstset{backgroundcolor=\color{lightGray},frame=single,basicstyle=\ttfamily}}{}

\newcommand{\orion}{\textsc{Or10n}\xspace}


%%%%%%%%%%%%%%%%%%%%%%%%%%%%%%%%%%%%%%%%%%%%%%%%%%%%%%%%%%%%%%%%%%%%%%%
%%%%%                                                                 %
%%%%%     Titlepage Macros - !!! DO NOT CHANGE !!!                    %
%%%%%                                                                 %
%%%%%%%%%%%%%%%%%%%%%%%%%%%%%%%%%%%%%%%%%%%%%%%%%%%%%%%%%%%%%%%%%%%%%%%
% Create a command for missing title page parameters.
\newcommand{\misspar}[1]{\textcolor{red}{\textbf{$<$#1$>$}}}

\makeatletter

% Redefine existing class macros as missing.
\title{\misspar{Specify Title}}%
\author{\misspar{Specify Author}}%
\date{\misspar{Specify Date}}%

% Define a command for setting the semester on the titlepage.
\def\@semester{\misspar{Specify Semester}}%
\newcommand{\setsemester}[1]{\def\@semester{#1}}%
\let\semester\setsemester%
\newcommand{\show@semester}{\@semester}%

% Define a command for setting the type of the report (Master Thesis,
% Semester Project, etc.) on the titlepage.
\def\@reporttype{\misspar{Specify Report Type}}%
\newcommand{\setreporttype}[1]{\def\@reporttype{#1}}%
\let\reporttype\setreporttype%
\newcommand{\show@reporttype}{\@reporttype}%

% Define a command for setting the image path for the image on the
% titlepage.
\def\@titlelogo{}%
\newcommand{\settitlelogo}[1]{\def\@titlelogo{#1}}%
\let\titlelogo\settitlelogo%

% Define a command for setting the image height on the titlepage.
\def\@logoheight{7cm}%
\newcommand{\setlogoheight}[1]{\def\@logoheight{#1}}%
\let\logoheight\setlogoheight%
\newcommand{\show@logoheight}{\@logoheight}%

% Define a command for setting the email on the titlepage.
\def\@email{\misspar{Specify E-Mail}}%
\newcommand{\setemail}[1]{\def\@email{#1}}%
\let\email\setemail%
\newcommand{\show@email}{\@email}%

% Define a command for setting the first supervisor on the titlepage.
\def\@firstsup{\misspar{Specify First Supervisor}}%
\newcommand{\setfirstsup}[1]{\def\@firstsup{#1}}%
\let\firstsup\setfirstsup%
\newcommand{\show@firstsup}{\@firstsup}%

% Define a command for setting the second supervisor on the titlepage.
\def\@secondsup{\misspar{Specify Second Supervisor}}%
\newcommand{\setsecondsup}[1]{\def\@secondsup{#1}}%
\let\secondsup\setsecondsup%
\newcommand{\show@secondsup}{\@secondsup}%

% Define a command for setting the professor on the titlepage.
\def\@professor{\misspar{Specify Professor}}%
\newcommand{\setprofessor}[1]{\def\@professor{#1}}%
\let\professor\setprofessor%
\newcommand{\show@professor}{\@professor}%

% Define a command for setting the margin on the title page.
\def\@titlepagemargin{3cm}%
\newcommand{\settitlepagemargin}[1]{\def\@titlepagemargin{#1}}%
\let\titlepagemargin\settitlepagemargin%
\newcommand{\show@titlepagemargin}{\@titlepagemargin}%

\makeatother

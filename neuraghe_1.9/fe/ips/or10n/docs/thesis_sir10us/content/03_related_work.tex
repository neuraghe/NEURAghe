%%%%%%%%%%%%%%%%%%%%%%%%%%%%%%%%%%%%%%%%%%%%%%%%%%%%%%%%%%%%%%%%%%%%%%%
%%%%%%%%%%%%%%%%%%%%%%%%%%%%%%%%%%%%%%%%%%%%%%%%%%%%%%%%%%%%%%%%%%%%%%%
%%%%%                                                                 %
%%%%%     03_related_work.tex                                         %
%%%%%                                                                 %
%%%%% Author:      Matthias Baer                                      %
%%%%% Created:     26/01/2014                                         %
%%%%% Description: Describing the OpenRISC project                    %
%%%%%                                                                 %
%%%%%%%%%%%%%%%%%%%%%%%%%%%%%%%%%%%%%%%%%%%%%%%%%%%%%%%%%%%%%%%%%%%%%%%
%%%%%%%%%%%%%%%%%%%%%%%%%%%%%%%%%%%%%%%%%%%%%%%%%%%%%%%%%%%%%%%%%%%%%%%

\chapter{Related Work}

\section{OpenRISC}

Our thesis is based on the OpenRISC 1000 project from OpenCores\cite{website:OpenCores}\cite{or1000}, which aims to create free and open source computing platforms. The architecture specifies behavior for 32- and 64-bit \gls{risc}/\glslink{dsp}{DSP} processors. It is designed for high performance, simplicity, low power consumption and scalability.

For the OpenRISC 1000 architecture exists an implementation called OpenRISC 1200 which is 32-bit with \gls{harvard}, 5 stage pipeline, virtual memory support and basic DSP capabilities (Figure~\ref{fig:or1200}). The implementation is written in Verilog and can be downloaded from the OpenCores website.\cite{or1200}

\begin{figure}[htbp]
  \centering
  \includegraphics[height=6cm]{./figures/Or1200_blocks}
  \caption{OpenRISC 1200 block diagram}
  \label{fig:or1200}
\end{figure}

In order to stay competitive, the OpenCores community called for OpenRISC \gls{asic} donations, they would use to create a "super-low-cost" \glslink{soc}{SoC} ASIC component, but up to the date this document is released, they were not yet successful.

The community has also ported the \gls{gnu} to OpenRISC, offering C and C++ support with static libraries. This allows us to create simple test programs for our implementation with quite low effort.\\
It also exists an architecture simulator capable of emulating OpenRISC based computer systems at the instruction level.

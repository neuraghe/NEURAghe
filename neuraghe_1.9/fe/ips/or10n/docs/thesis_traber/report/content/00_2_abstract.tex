%%%%%%%%%%%%%%%%%%%%%%%%%%%%%%%%%%%%%%%%%%%%%%%%%%%%%%%%%%%%%%%%%%%%%%%
%%%%%%%%%%%%%%%%%%%%%%%%%%%%%%%%%%%%%%%%%%%%%%%%%%%%%%%%%%%%%%%%%%%%%%%
%%%%%                                                                 %
%%%%%     00_2_abstract.tex                                           %
%%%%%                                                                 %
%%%%% Author:      <author>                                           %
%%%%% Created:     <date>                                             %
%%%%% Description: <description>                                      %
%%%%%                                                                 %
%%%%%%%%%%%%%%%%%%%%%%%%%%%%%%%%%%%%%%%%%%%%%%%%%%%%%%%%%%%%%%%%%%%%%%%
%%%%%%%%%%%%%%%%%%%%%%%%%%%%%%%%%%%%%%%%%%%%%%%%%%%%%%%%%%%%%%%%%%%%%%%

\chapter*{Abstract}
Today's embedded devices like wearables, smartphones, Internet of Things
devices and sensors need a vast amount of computing power in a very constrained
environment where only a limited amount of energy is available.
By reducing the operating voltage of digital circuits to near-threshold values,
the energy efficiency of those circuits can be improved. To recover the loss in
speed, parallelization can be employed.

In our Parallel Ultra-Low power Processor (PULP), several OpenRISC based cores
are organized in clusters to perform computations in parallel. To increase
their energy efficiency at low voltages even further, the instruction set of
those cores was extended by adding vectorial instructions, bit counting
operations and improvements to the MAC unit. In previous work, extensions for
hardware loops and auto-incrementing load and store instructions were added to
the core on which the new instructions are built on. With those extensions, the
cores are able to perform more computations per cycle and thus need to stay
active for a shorter period of time. At the same time the core area has only
increased by $25\%$, while the area of one PULP cluster has increased by $2\%$
due to our additions.
The critical path delay of the core was unaffected by the extensions.

Compared to the original OpenRISC instruction set, a performance gain of up to
a factor of 5x was achieved. In terms of energy efficiency we were able to be
$45\%$ more energy efficient on average.

